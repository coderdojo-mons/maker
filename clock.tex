\documentclass[11pt,a4paper,oneside,twocolumn]{article}
%\usepackage{fourier} %fonte plus lisible
\usepackage{hyperref}
\usepackage[T1]{fontenc}
\usepackage[francais]{babel}
\usepackage[utf8]{inputenc}
\usepackage{cite}
%\usepackage{lipsum}
\usepackage{amsmath}
\usepackage{amsthm}
\usepackage{amssymb}
\usepackage{verbatim}
%\usepackage{varwidth}
\usepackage{pgf,tikz}
\usetikzlibrary{calc,arrows}
\usepackage{url}
\usepackage{subcaption}
\usepackage[siunitx]{circuitikz}
\usepackage{amsmath}
\usepackage{siunitx}
\usepackage{listings}
\usepackage{caption}
\DeclareCaptionType{equ}[\textsc{Équation}]{}

\title{Conception d'un circuit d'horloge}
\author{Guillaume \textsc{Huysmans}, Ben \textsc{Eater}}
\hypersetup{pdfauthor={Guillaume Huysmans, Ben Eater},
	pdftitle={Conception d'un circuit d'horloge},
	pdfsubject={électronique, temps, bistable},
	pdfkeywords={électronique, physique, circuit}}

\begin{document}
\maketitle

\makeatletter
\pgfdeclareshape{srl}{
  % The 'minimum width' and 'minimum height' keys, not the content, determine
  % the size
  \savedanchor\northeast{%
    \pgfmathsetlength\pgf@x{\pgfshapeminwidth}%
    \pgfmathsetlength\pgf@y{\pgfshapeminheight}%
    \pgf@x=0.5\pgf@x
    \pgf@y=0.5\pgf@y
  }
  % This is redundant, but makes some things easier:
  \savedanchor\southwest{%
    \pgfmathsetlength\pgf@x{\pgfshapeminwidth}%
    \pgfmathsetlength\pgf@y{\pgfshapeminheight}%
    \pgf@x=-0.5\pgf@x
    \pgf@y=-0.5\pgf@y
  }
  % Inherit from rectangle
  \inheritanchorborder[from=rectangle]

  % Define same anchor a normal rectangle has
  \anchor{center}{\pgfpointorigin}
  \anchor{north}{\northeast \pgf@x=0pt}
  \anchor{east}{\northeast \pgf@y=0pt}
  \anchor{south}{\southwest \pgf@x=0pt}
  \anchor{west}{\southwest \pgf@y=0pt}
  \anchor{north east}{\northeast}
  \anchor{north west}{\northeast \pgf@x=-\pgf@x}
  \anchor{south west}{\southwest}
  \anchor{south east}{\southwest \pgf@x=-\pgf@x}
  \anchor{text}{
    \pgfpointorigin
    \advance\pgf@x by -.5\wd\pgfnodeparttextbox%
    \advance\pgf@y by -.5\ht\pgfnodeparttextbox%
    \advance\pgf@y by +.5\dp\pgfnodeparttextbox%
  }

  % Define anchors for signal ports
  \anchor{R}{
    \pgf@process{\northeast}%
    \pgf@x=-1\pgf@x%
    \pgf@y=.5\pgf@y%
  }
  \anchor{S}{
    \pgf@process{\northeast}%
    \pgf@x=-1\pgf@x%
    \pgf@y=-.5\pgf@y%
  }
  \anchor{Q}{
    \pgf@process{\northeast}%
    \pgf@y=.5\pgf@y%
  }
  \anchor{Qn}{
    \pgf@process{\northeast}%
    \pgf@y=-.5\pgf@y%
  }
  % Draw the rectangle box and the port labels
  \backgroundpath{
    % Rectangle box
    \pgfpathrectanglecorners{\southwest}{\northeast}
    \pgfclosepath

    % Draw port labels
    \begingroup
    \tikzset{flip flop/port labels} % Use font from this style
    \tikz@textfont

    \pgf@anchor@srl@R
    \pgftext[left,base,at={\pgfpoint{\pgf@x}{\pgf@y}},x=\pgfshapeinnerxsep]{\raisebox{-0.75ex}{R}}

    \pgf@anchor@srl@S
    \pgftext[left,base,at={\pgfpoint{\pgf@x}{\pgf@y}},x=\pgfshapeinnerxsep]{\raisebox{-0.75ex}{S}}

    \pgf@anchor@srl@Q
    \pgftext[right,base,at={\pgfpoint{\pgf@x}{\pgf@y}},x=-\pgfshapeinnerxsep]{\raisebox{-.75ex}{Q}}

    \pgf@anchor@srl@Qn
    \pgftext[right,base,at={\pgfpoint{\pgf@x}{\pgf@y}},x=-\pgfshapeinnerxsep]{\raisebox{-.75ex}{$\overline{\mbox{Q}}$}}
    \endgroup
  }
}

% Key to add font macros to the current font
\tikzset{add font/.code={\expandafter\def\expandafter\tikz@textfont\expandafter{\tikz@textfont#1}}} 

% Define default style for this node
\tikzset{flip flop/port labels/.style={font=\sffamily\scriptsize}}
\tikzset{every srl node/.style={draw,minimum width=2cm,minimum 
height=2.828427125cm,very thick,inner sep=1mm,outer sep=0pt,cap=round,add 
font=\sffamily}}

\makeatother

\shorthandoff{:!}

%adapté http://www.texample.net/tikz/examples/d-flip-flops-and-shift-register/
%idée aussi dans https://tex.stackexchange.com/a/240748 : node, pas rectangle !
\tikzset{empty/.style = {inner sep=0, outer sep=0}}
\begin{figure}[ht]
	\centering
	\begin{circuitikz}\draw %[scale=0.8]
		(5,5.5) node[op amp, yscale=-1](cmp_r){}
		(cmp_r)++(0,-3) node[op amp, yscale=-1](cmp_s){}
		(cmp_s)++(3,1.5) node[srl](sr){}
		(cmp_r.out) |- (sr.R)
		(cmp_s.out) |- (sr.S)
		(2,-0.5) node[ground](gnd){}
		to[R] ++(0,3) node[empty](one_third){}
		to[R, *-] ++(0,2) node[empty](two_thirds){}
		to[R, *-] ++(0,3)
		node[vcc]{}
		(one_third) -| (cmp_s.+)
		(two_thirds) to ++(1,0) node[empty](ctl){} -| (cmp_r.-) %FIXME *
		%FIXME align them!
		(cmp_s.-) to ++(-4,0) node[left]{Trigger}
		(cmp_r.+) |- ++(-4,1) node[left]{Threshold}
		(cmp_r.out)++(0,2) node[not port, rotate=-90](i){}
		(cmp_r.out) |- (i.out) %FIXME *
		(i.in) |- ++(-6,1) node[left]{Reset} %FIXME *
		(ctl) |- ++(-3,0.5) node[left]{Control}
		(sr.Q) to ++(2,0) node[right]{Output}
		(sr.Qn)++(1,-3.5) node[npn, rotate=-90](dis){}
		(sr.Qn) -| (dis.B)
		(dis.C) to ++(0.5,0) node[right]{Discharge}
		(dis.E) -| (gnd) %FIXME *
		;
	\end{circuitikz}
	\caption{Circuit 555}
	\label{fig:ser}
\end{figure}

%expliquer comment on utilise un ampli ici
%donner les différents modes de fonctionnement : monostable, bistable, astable
%expliquer les maths derrière la charge ? non, citer circuit RC

\end{document}
